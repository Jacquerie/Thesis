\chapter{Introduzione e definizioni preliminari}
Il concetto di entropia di grafo fu introdotto da Janos K\"oner come possibile risposta al problema di assegnare ad un grafo un numero che ne rappresenti la complessità, ma che allo stesso tempo ammetta un'interpretazione in termini di teoria dell'informazione.
Questo si ottiene interpretando il grafo come rappresentante la relazione di distinguibilità dei simboli emessi da una sorgente discreta. Più precisamente, consideriamo $X$ una sorgente la quale ad ogni istante discreto emetta un simbolo $v_i$ dall'alfabeto finito $\{v_1,\dots,v_n\}$, con probabilità $p_1,\dots,p_n$ rispettivamente, ma supponiamo di non essere in grado di distinguere $v_i$ da $v_j$ per ogni $i,j$. Rappresentiamo dunque tale relazione di distinguibilità con un grafo con vertici i simboli $v_1,\dots,v_n$ e con un arco $(v_i,v_j)$ ogni volta che i simboli $v_i$ e $v_j$ sono distinguibili; questa relazione, come è immediato verificare, è simmetrica ma non necessariamente transitiva nè riflessiva. 
Vogliamo inoltre che, una volta definita, l'entropia sia, per un grafo completo (ovvero nel caso di totale distinguibilità), equivalente alla classica nozione di entropia per una sorgente; mostreremo infatti che essa generalizza l'entropia di Shannon.

Grazie ad alcune sue proprietà peculiari, ad esempio una particolare forma di subadditività, l'entropia di grafo ha trovato applicazione in campo combinatorico e algoritmico, in particolare per dimostrare certe disuguaglianze.
Lo stesso K\"orner l'ha usata per mostrare nuovamente una stima sul numero di funzioni di hash perfetto di un insieme dovuta a Fredman e Kolmos e successivamente, insieme a Marton, per migliorare tale stima sfruttando le disuguaglianze più precise che si ottengono estendendo il concetto di entropia agli ipergrafi. 
Sempre K\"orner, in collaborazione con altri, ha analizzato i casi in cui la subbaditività è in realtà un'uguaglianza, deducendone una caratterizzazione dei grafi perfetti in termini di entropia e quindi una nuova dimostrazione della congettura debole che li riguarda.
Kahn e Kim hanno invece esibito un algoritmo che utilizza l'entropia per determinare additivamente un ordine totale fissato in precedenza ma ignoto, tramite il calcolo della stessa su una particolare successione di grafi associati. Questo approccio ha anche consentito di stimare il numero di estensioni lineari di un dato ordine parziale, la cui determinazione esatta è un problema \#P-completo.

Per tutto il seguito con $G=(V,E)$ indicheremo un grafo semplice non diretto di insieme di vertici $V$ ed insieme degli archi $E$. Ricordiamo inoltre che un sottoinsieme di vertici di $G$ è un \emph{insieme indipendente} se essi sono due a due non adiacenti e che una \emph{colorazione} di $G$ è una partizione di $V$ in insiemi indipendenti. Definiamo quindi il \emph{numero cromatico} di $G$ il numero minimo $\chi(G)$ di insiemi indipendenti necessari a colorare $G$. Infine, assumiamo che tutti i logaritmi siano in base $2$.

La rigorosa definizione di entropia di grafo comporta alcune difficoltà tecniche, la cui risoluzione costituisce l'obiettivo principale del primo capitolo. Nel secondo capitolo verranno enunciate e dimostrate alcune proprietà di rilievo dell'entropia.
