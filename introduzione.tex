\chapter{Introduzione} 
Il concetto di entropia di una sorgente fu introdotto da Claude Shannon nel fondamentale articolo ``A Mathematical Theory of Communication'' del 1948 \cite{Shannon1948}.
\begin{definition}
  Sia \(X\) una sorgente che a ogni istante discreto emetta un simbolo \(v_i\) nell'alfabeto finito \(\{v_1,\dots,v_n\}\) con probabilit\`a \(p_i\) per \(1\le i\le n\). Chiamiamo \emph{entropia di \(X\)} la quantit\`a
  \[H(X)=\sum_{i=1}^n{p_i\log{\frac{1}{p_i}}}\text{.}\]
\end{definition}
Dalla sua introduzione sono state proposte pi\`u generalizzazioni, fra cui ricordiamo l'entropia di R\'enyi TODO:citazione e TODO. In questa tesi andremo a esporre un'ulteriore generalizzazione dovuta a Janos K\"orner, nota come ``Entropia di grafo''. Infatti, oltre a generalizzare la nozione di entropia di una sorgente, questa entropia consente di assegnare a un grafo un numero che ne rappresenti la complessit\`a. Ci\`o si ottiene interpretando il grafo come rappresentante la relazione di distinguibilit\`a dei simboli emessi da una sorgente discreta. Pi\`u precisamente, detta \(X\) una sorgente come nella precedente definizione, associamo a essa un grafo di insieme dei vertici \(V=\{v_1,\dots,v_n\}\) e un arco \((v_i,v_j)\) ogni volta che i simboli \(v_i\) e \(v_j\) sono distinguibili. La definizione rigorosa di entropia di grafo comporta alcune difficolt\`a tecniche, la cui risoluzione costituisce l'obiettivo del primo capitolo. Nel secondo capitolo passeremo a esporre le principali propriet\`a dell'entropia di grafo, fra cui la monotonia e una forma di subadditivit\`a. Enunceremo inoltre l'interessante relazione con i grafi perfetti, cio\`e quei grafi per cui numero cromatico e numero di cricca coincidono. Ne dedurremo alcuni lemmi sui grafi di confrontabilit\`a associati agli insiemi parzialmente ordinati. Questi risultati verranno sfruttati nel terzo capitolo per descrivere tre algoritmi che risolvano il problema dell'ordinamento con informazione parziale. Tale problema consiste nel determinare adattivamente un ordine totale fissato ma ignoto a partire da un insieme parzialmente ordinato. Jeff Kahn e Jeong Kim nell'articolo ``Entropy and Sorting'' del 1995 \cite{Kahn1995} evidenziarono per primi il collegamento esistente fra il problema dell'ordinamento con informazione parziale e l'entropia del grafo a esso associato. Essi esibirono inoltre un algoritmo per la soluzione di tale problema in un numero asintoticamente ottimo di confronti e per giunta polinomiale nelle operazioni elementari, ma che a ogni passo fa uso del metodo dell'ellissoide. In un recente articolo Jean Cardinal et al. hanno invece esibito tre algoritmi per la soluzione dello stesso problema in un numero asintoticamente ottimo di confronti e comunque polinomiali nelle operazioni elementari, senza far uso del metodo dell'ellissoide \cite{Cardinal2010}. Nel terzo capitolo verranno illustrati in dettaglio tali algoritmi.

Sull'entropia di grafo \`e stata scritta una esauriente survey da Gabor Simonyi nel 1995 \cite{Simonyi1995}, e una versione pi\`u aggiornata nel 2001 \cite{Simonyi2001}. Pi\`u in generale, Moshowitz e TODO hanno raccolto le varie nozioni di complessit\`a di un grafo che sono state proposte, spesso denominate ``entropia di grafo'' TODO:citazione. (TODO: entropia di K\"orner?)

