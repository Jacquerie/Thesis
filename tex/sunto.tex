\documentclass[12pt]{article}

\usepackage[italian]{babel}

% Use utf-8 encoding for foreign characters
\usepackage[utf8]{inputenc}

% Setup for fullpage use
\usepackage{fullpage}

% More symbols
\usepackage{amsmath} 
\usepackage{amssymb} 
\usepackage{latexsym} 
\usepackage{enumerate}

\newenvironment{definition}[1][Definizione.]{ 
\begin{trivlist}
	\item[\hskip \labelsep {\bfseries #1}]}{ 
\end{trivlist}
}

% Pretty bibliography with biblatex
\usepackage[style=numeric,maxcitenames=5,maxbibnames=5,backend=biber]{biblatex} 
\bibliography{tesi}

\begin{document}

\section*{Entropia di grafo e il problema dell'ordinamento con informazione parziale}
  
Il concetto di entropia di una sorgente fu introdotto da Claude E. Shannon nel fondamentale articolo ``A Mathematical Theory of Communication'' \cite{Shannon1948}.
\begin{definition}
  Sia \(X\) una sorgente che a ogni istante discreto emetta un simbolo \(v_i\) nell'alfabeto finito \(\{v_1,\dots,v_n\}\) con probabilità \(p_i\) per \(1\le i\le n\). Chiamiamo \emph{entropia di \(X\)} la quantità
  \[H(X)=\sum_{i=1}^n{p_i\log{\frac{1}{p_i}}}\text{.}\]
\end{definition}

Dalla sua introduzione sono state proposte più generalizzazioni, fra cui 
ricordiamo in particolare l'entropia di Rényi \cite{Renyi1961}. In questa tesi 
andremo a esporre un'ulteriore generalizzazione dovuta a Janos K\"orner, nota 
come ``entropia di grafo'' \cite{Korner1973}. Infatti, oltre a generalizzare 
la nozione di entropia di una sorgente, questa entropia consente di assegnare 
a un grafo un numero che ne rappresenti la complessità. Ciò si ottiene 
interpretando il grafo come rappresentante la relazione di distinguibilità dei 
simboli emessi da una sorgente discreta. Più precisamente, detta \(X\) una 
sorgente come nella precedente definizione, associamo a essa un grafo di 
insieme dei vertici \(V=\{v_1,\dots,v_n\}\) e un arco \((v_i,v_j)\) ogni volta 
che i simboli \(v_i\) e \(v_j\) sono distinguibili. Vorremmo quindi riottenere 
la classica nozione di entropia di una sorgente nel caso di totale 
distinguibilità, ovvero di grafo completo.

La definizione rigorosa di entropia di grafo comporta alcune difficoltà 
tecniche, la cui soluzione costituisce l'obiettivo del primo capitolo. Nel 
secondo capitolo passeremo a esporre le principali proprietà dell'entropia di 
grafo, fra cui la monotonia e una forma di subadditività. Enunceremo inoltre 
l'interessante relazione con i grafi perfetti, cioè quei grafi per cui numero 
cromatico e numero di cricca coincidono. Ne dedurremo alcuni lemmi sui grafi 
di confrontabilità associati agli insiemi parzialmente ordinati. Questi 
risultati verranno sfruttati nel terzo capitolo per descrivere algoritmi che 
risolvano il problema dell'ordinamento con informazione parziale. Tale 
problema consiste nel determinare adattivamente un ordine totale fissato ma 
ignoto a partire da un insieme parzialmente ordinato. Jeff Kahn e Jeong H. Kim 
nell'articolo ``Entropy and Sorting'' evidenziarono per primi il collegamento 
esistente fra il problema dell'ordinamento con informazione parziale e 
l'entropia del grafo a esso associato \cite{Kahn1995}. Essi esibirono inoltre 
un algoritmo per la soluzione di tale problema in un numero asintoticamente 
ottimo di confronti e per giunta polinomiale nelle operazioni elementari, ma 
che a ogni passo fa uso del metodo dell'ellissoide. In un recente articolo 
Jean Cardinal et al. hanno invece esibito tre algoritmi per la soluzione dello 
stesso problema in un numero asintoticamente ottimo di confronti e comunque 
polinomiali nelle operazioni elementari, senza però far uso del metodo 
dell'ellissoide \cite{Cardinal2010}. Nel terzo capitolo verranno illustrati in 
dettaglio questi ultimi tre algoritmi.

Sull'entropia di grafo è stata scritta un'esauriente rassegna da G\'abor 
Simonyi nel 1995 e una versione più aggiornata nel 2001 \cite{Simonyi1995}, 
\cite{Simonyi2001}.

\printbibliography

\end{document}
