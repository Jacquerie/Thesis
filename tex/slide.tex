\documentclass[10pt]{beamer}

% Nice accents and good sillabation
\usepackage[italian]{babel}

% Use utf-8 encoding for foreign characters
\usepackage[utf8]{inputenc}

% Pretty pictures with TikZ
\usepackage{tikz}
\usetikzlibrary{shapes}
\tikzstyle{red_dot}=[circle, color=red!100, fill=red!100, inner sep=0.pt, minimum size=0.2cm, draw]
\tikzstyle{vertex}=[circle, draw=black!100, thick, minimum size=0.5cm]
\tikzstyle{red_vertex}=[circle, draw=red!100, thick, minimum size=0.5cm]
\tikzstyle{pale_vertex}=[circle, draw=black!25!white, thick, minimum size=0.5cm]
\tikzstyle{normal_ellipse}=[ellipse, minimum width=2.5cm, minimum height=1.1cm, color=red, thick, draw]
\tikzstyle{pale_ellipse}=[ellipse, minimum width=2.5cm, minimum height=1.1cm, color=red!40!white, thick, draw]

% Pretty algorithms with algorithm and algorithmic
\usepackage{algorithm}
\usepackage{algorithmic}
% Hacks needed to get the algorithm package to work.
\renewcommand{\algorithmiccomment}[1]{/\(\!\)/ #1} \floatname{algorithm}{Algoritmo}

\usetheme{Copenhagen}

\newtheorem{teorema}{Teorema} 
\newtheorem{definizione}{Definizione} 
\newtheorem{congettura}{Congettura} 
\newtheorem{osservazione}{Osservazione}
\newtheorem{proposizione}{Proposizione}
\newtheorem{dimostrazione}{Dimostrazione}

\AtBeginSection[]
{
  \begin{frame}
    \tableofcontents[currentsection]
  \end{frame}
}

\begin{document}

\title[Ordinamento con informazione parziale]{Entropia di grafo e il problema dell'ordinamento con informazione parziale} 
\author{Jacopo Notarstefano} 
\date{20 Luglio 2012}

\begin{frame}[plain]
  \titlepage
\end{frame}

\begin{frame}
  \tableofcontents
\end{frame}

\section{Il problema dell'ordinamento con informazione parziale}
\begin{frame}
  {Il problema dell'ordinamento con informazione parziale}
  \begin{definizione}
    Sia \(P=(V,\le_P)\) un insieme parzialmente ordinato. Diciamo che un 
    ordine totale \(<\) \`e un'\emph{estensione lineare} di \(\le_P\) se, 
    \(\forall v_i, v_j\in V\text{,}\)
    \[v_i\le_P v_j \implies v_i < v_j\text{.}\]
    Denotiamo inoltre con \(e(P)\) il numero di estensioni lineari di \(P\). 
  \end{definizione}
  \vspace{5mm}
  \begin{definizione}
    Sia \(P=(V,\le_P)\) un insieme parzialmente ordinato. Il                     
    \emph{problema dell'ordinamento con informazione parziale} consiste nel 
    determinare un'estensione lineare \(<\) fissata ma ignota per mezzo di 
    domande del tipo ``\`E vero che \(v_i < v_j\)?'', detti \emph{confronti}. 
  \end{definizione}
\end{frame}

\begin{frame}
  {Esempio di ordinamento con informazione parziale}
  \begin{figure}
    \centering
    \begin{tikzpicture}
      \node at (0,0) (a1) [vertex] {\(a\)};
      \node at (0,3/2) (b1) [vertex] {\(b\)};
      \node at (3/4, 3) (c1) [vertex] {\(c\)};
      \node at (3/2,3/2) (d1) [vertex] {\(d\)};
      
      \draw [->] (a1) -- (b1);
      \draw [->] (b1) -- (c1);
      \draw [->] (d1) -- (c1);

      \path[use as bounding box] (-1,-1) rectangle (10,4);
    \end{tikzpicture}
  \end{figure}
\end{frame}

\begin{frame}
  {Esempio di ordinamento con informazione parziale}
  \begin{figure}
    \centering
    \begin{tikzpicture}
      \node at (0,0) (a1) [vertex] {\(a\)};
      \node at (0,3/2) (b1) [vertex] {\(b\)};
      \node at (3/4, 3) (c1) [vertex] {\(c\)};
      \node at (3/2,3/2) (d1) [vertex] {\(d\)};
      
      \node at (3/4,3/2) [normal_ellipse] {};
      
      \draw [->] (a1) -- (b1);
      \draw [->] (b1) -- (c1);
      \draw [->] (d1) -- (c1);

      \path[use as bounding box] (-1,-1) rectangle (10,4);
    \end{tikzpicture}
  \end{figure}
\end{frame}

\begin{frame}
  {Esempio di ordinamento con informazione parziale}
  \begin{figure}
    \centering
    \begin{tikzpicture}
      \node at (0,0) (a1) [vertex] {\(a\)};
      \node at (0,3/2) (b1) [vertex] {\(b\)};
      \node at (3/4, 3) (c1) [vertex] {\(c\)};
      \node at (3/2,3/2) (d1) [vertex] {\(d\)};
      
      \node at (3/4,3/2) [normal_ellipse] {};
      
      \draw [->] (a1) -- (b1);
      \draw [->] (b1) -- (c1);
      \draw [->] (d1) -- (c1);
      
      \node at (3,3/2) [] {\(\longrightarrow\)};
      
      \node at (4,0) (a2) [vertex] {\(a\)};
      \node at (19/4,3/2) (b2) [vertex] {\(b\)};
      \node at (19/4, 3) (c2) [vertex] {\(c\)};
      \node at (11/2,0) (d2) [vertex] {\(d\)};

      \draw [->] (a2) -- (b2);
      \draw [->] (b2) -- (c2);
      \draw [->] (d2) -- (b2);
      
      \path[use as bounding box] (-1,-1) rectangle (10,4);
    \end{tikzpicture}
  \end{figure}
\end{frame}

\begin{frame}
  {Esempio di ordinamento con informazione parziale}
  \begin{figure}
    \centering
    \begin{tikzpicture}
      \node at (0,0) (a1) [pale_vertex] {\(a\)};
      \node at (0,3/2) (b1) [pale_vertex] {\(b\)};
      \node at (3/4, 3) (c1) [pale_vertex] {\(c\)};
      \node at (3/2,3/2) (d1) [pale_vertex] {\(d\)};
      
      \node at (3/4,3/2) [pale_ellipse] {};
      
      \draw [->, color=black!25!white] (a1) -- (b1);
      \draw [->, color=black!25!white] (b1) -- (c1);
      \draw [->, color=black!25!white] (d1) -- (c1);
      
      \node at (4,0) (a2) [vertex] {\(a\)};
      \node at (19/4,3/2) (b2) [vertex] {\(b\)};
      \node at (19/4, 3) (c2) [vertex] {\(c\)};
      \node at (11/2,0) (d2) [vertex] {\(d\)};
      
      \node at (19/4, 0) [normal_ellipse] {};
      
      \draw [->] (a2) -- (b2);
      \draw [->] (b2) -- (c2);
      \draw [->] (d2) -- (b2);

      \path[use as bounding box] (-1,-1) rectangle (10,4);
    \end{tikzpicture}
  \end{figure}
\end{frame}

\begin{frame}
  {Esempio di ordinamento con informazione parziale}
  \begin{figure}
    \centering
    \begin{tikzpicture}
      \node at (0,0) (a1) [pale_vertex] {\(a\)};
      \node at (0,3/2) (b1) [pale_vertex] {\(b\)};
      \node at (3/4, 3) (c1) [pale_vertex] {\(c\)};
      \node at (3/2,3/2) (d1) [pale_vertex] {\(d\)};
      
      \node at (3/4,3/2) [pale_ellipse] {};
      
      \draw [->, color=black!25!white] (a1) -- (b1);
      \draw [->, color=black!25!white] (b1) -- (c1);
      \draw [->, color=black!25!white] (d1) -- (c1);
      
      \node at (4,0) (a2) [vertex] {\(a\)};
      \node at (19/4,3/2) (b2) [vertex] {\(b\)};
      \node at (19/4, 3) (c2) [vertex] {\(c\)};
      \node at (11/2,0) (d2) [vertex] {\(d\)};
      
      \node at (19/4, 0) [normal_ellipse] {};
      
      \draw [->] (a2) -- (b2);
      \draw [->] (b2) -- (c2);
      \draw [->] (d2) -- (b2);
      
      \node at (6.5,3/2) [] {\(\longrightarrow\)};
      
      \node at (33/4,0) (a3) [vertex] {\(a\)};
      \node at (33/4,1) (d3) [vertex] {\(d\)};
      \node at (33/4,2) (b3) [vertex] {\(b\)};
      \node at (33/4,3) (c3) [vertex] {\(c\)};
      
      \draw [->] (a3) -- (d3);
      \draw [->] (d3) -- (b3);
      \draw [->] (b3) -- (c3);
      
      \path[use as bounding box] (-1,-1) rectangle (10,4);
    \end{tikzpicture}
  \end{figure}
\end{frame}

\begin{frame}
  {Stato dell'arte}
  Sia \(P\) un insieme parzialmente ordinato di cardinalità \(n\). Sono necessari
  \(\Omega(\log_2{e(P)})\) confronti affinch\'e un algoritmo risolutivo sia
  corretto.
  \vspace{5mm}
  \begin{table}
    \centering
    \begin{tabular}{ r | l l }
      \multicolumn{1}{r}{}
      & \multicolumn{1}{l}{\#Confronti}
      & \multicolumn{1}{l}{Complessità} \\
      \cline{2-3}
      Fredman 1976 & \(\log{e(P)} + 2n\) & superpolinomiale\\
      Kahn \& Saks 1984 & \(O(\log{e(P)})\) & superpolinomiale\\
      Kahn \& Kim 1995 & \(O(\log{e(P)})\) & polinomiale*\\
      Cardinal et al. 2010 & \(O(\log{e(P)})\) & polinomiale\\
    \end{tabular}
  \end{table}
  \vspace{5mm}
  * usa a ogni passo il metodo dell'ellissoide.
\end{frame}

\section{Entropia di grafo}
\begin{frame}
  {Insiemi indipendenti e politopo dei vertici}
  \begin{definizione}
    Sia \(G=(V,E)\) un grafo. Chiamiamo \emph{insieme indipendente} (o 
    \emph{stabile}) un sottoinsieme dei vertici \(W\) tale che il sottografo 
    indotto \(G(W)\) sia vuoto.
  \end{definizione}
  \vspace{5mm}
  \begin{definizione}
    Sia \(G\) un grafo. Chiamiamo \emph{politopo dei vertici} l'involucro 
    convesso \(\text{STAB}(G)\) dei vettori caratteristici degli insiemi 
    indipendenti, cioè:
    \[\text{STAB}(G) = \text{conv}\left\{\chi^S\in\{0,1\}^V\mid S\subset V,\, S\;\text{insieme indipendente}\right\}\]
  \end{definizione}
\end{frame}

\begin{frame}
  {Esempio di politopo dei vertici}
  \begin{figure}
    \centering
    \begin{tikzpicture}
      \node (b) at (0,0) [vertex] {b}; 
      \node (a) at (2,0) [vertex] {a}; 
      \node (c) at (1,1.732) [vertex] {c};
  
      \draw (b.0) -- (a.180) [thick]; 
      \draw (b.60) -- (c.240) [thick];
      
      \draw [->] (8,0,0) -- (8.5,0,0) node [right] {\(x\)};
      \draw [->] (6,2,0) -- (6,2.5,0) node [above] {\(y\)};
      \draw [->] (6,0,2) -- (6,0,3) node [below left] {\(z\)};
      
      \draw (8,0,0) -- (6,2,2) [thick]; 
      \draw (6,2,0) -- (6,2,2) [thick]; 
      \draw (6,0,2) -- (6,2,2) [thick]; 
      \draw (8,0,0) -- (6,0,2) [thick]; 
      \draw (8,0,0) -- (6,2,0) [thick]; 
      \draw (6,0,0) -- (8,0,0) [thick, dashed]; 
      \draw (6,0,0) -- (6,2,0) [thick, dashed]; 
      \draw (6,0,0) -- (6,0,2) [thick, dashed];
    \end{tikzpicture}
  \end{figure}
\end{frame}

\begin{frame}
  {Esempio di politopo dei vertici}
  \begin{figure}
    \centering
    \begin{tikzpicture}
      \node (b) at (0,0) [red_vertex] {b}; 
      \node (a) at (2,0) [vertex] {a}; 
      \node (c) at (1,1.732) [vertex] {c};
  
      \draw (b.0) -- (a.180) [thick]; 
      \draw (b.60) -- (c.240) [thick];
      
      \draw [->] (8,0,0) -- (8.5,0,0) node [right] {\(x\)};
      \draw [->] (6,2,0) -- (6,2.5,0) node [above] {\(y\)};
      \draw [->] (6,0,2) -- (6,0,3) node [below left] {\(z\)};
      
      \draw (8,0,0) -- (6,2,2) [thick]; 
      \draw (6,2,0) -- (6,2,2) [thick]; 
      \draw (6,0,2) -- (6,2,2) [thick]; 
      \draw (8,0,0) -- (6,0,2) [thick]; 
      \draw (8,0,0) -- (6,2,0) [thick]; 
      \draw (6,0,0) -- (8,0,0) [thick, dashed]; 
      \draw (6,0,0) -- (6,2,0) [thick, dashed]; 
      \draw (6,0,0) -- (6,0,2) [thick, dashed];
      
      \node at (8,0,0) [red_dot] {};
    \end{tikzpicture}
  \end{figure}
\end{frame}

\begin{frame}
  {Esempio di politopo dei vertici}
  \begin{figure}
    \centering
    \begin{tikzpicture}
      \node (b) at (0,0) [vertex] {b}; 
      \node (a) at (2,0) [red_vertex] {a}; 
      \node (c) at (1,1.732) [red_vertex] {c};
  
      \draw (b.0) -- (a.180) [thick]; 
      \draw (b.60) -- (c.240) [thick];
      
      \draw [->] (8,0,0) -- (8.5,0,0) node [right] {\(x\)};
      \draw [->] (6,2,0) -- (6,2.5,0) node [above] {\(y\)};
      \draw [->] (6,0,2) -- (6,0,3) node [below left] {\(z\)};
            
      \draw (8,0,0) -- (6,2,2) [thick]; 
      \draw (6,2,0) -- (6,2,2) [thick]; 
      \draw (6,0,2) -- (6,2,2) [thick]; 
      \draw (8,0,0) -- (6,0,2) [thick]; 
      \draw (8,0,0) -- (6,2,0) [thick]; 
      \draw (6,0,0) -- (8,0,0) [thick, dashed]; 
      \draw (6,0,0) -- (6,2,0) [thick, dashed]; 
      \draw (6,0,0) -- (6,0,2) [thick, dashed];
      
      \node at (6,2,2) [red_dot] {};
    \end{tikzpicture}
  \end{figure}
\end{frame}

\begin{frame}
  {Definizione di entropia di grafo}
  \begin{definizione}
    Sia \(G\) un grafo e \(\text{STAB}(G)\) il politopo dei vertici di 
    \(G\). Chiamiamo \emph{entropia di grafo} il seguente minimo:
    \[
      H(G)=\min_{\substack{\mathbf{a}\in\text{STAB}(G) \\ \mathbf{a}>0}}{\frac{1}{n}\sum_{i=1}^n{\log{\frac{1}{a_i}}}}
    \]
  \end{definizione}
\end{frame}

\begin{frame}
  {Storia dell'entropia di grafo}
  \begin{itemize}
  \item Definizione originale di K\"orner (1973), motivata da un
    problema di teoria dell'informazione.
  \item Ne sono state date più definizioni equivalenti dall'aspetto
    molto diverso e sono note varie applicazioni in ambito algoritmico
    e combinatoriale.
  \item K\"orner e Marton (1988) hanno dato una stima pi\`u precisa
    del numero di funzioni di hash perfetto di un insieme.
  \item Esiste un interessante collegamento con la teoria dei grafi
    perfetti di Berge (1961).
  \end{itemize}
\end{frame}

\begin{frame}
  {Grafi perfetti}
  \begin{definizione}
    Sia \(G\) un grafo. Chiamiamo \emph{cricca} un sottografo completo e massimale,
    \emph{numero di cricca} \(\omega(G)\) la massima cardinalit\`a di una cricca.
  \end{definizione}
  \vspace{5mm}
  \`E ovvio che \(\omega(G)\le\chi(G)\). Per quali grafi vale l'uguaglianza?
  \vspace{5mm}
  \begin{definizione}[Berge 1961]
    Sia \(G\) un grafo. Diciamo che \(G\) \`e perfetto se, per ogni sottografo indotto
    \(H\), si ha \(\omega(H)=\chi(H)\).
  \end{definizione}
\end{frame}

\begin{frame}
  {Entropia e grafi perfetti}
  \begin{teorema}[Csiszár, K\"orner, Lovász, Marton, Simonyi 1990]
    Sia \(G\) un grafo. Se \(G\) \`e perfetto allora
    \[
      H(G) + H(\overline{G}) = \log{n}\text{.}
    \]
  \end{teorema}
  \vspace{3mm}
  Da questo \`e possibile dedurre il seguente
  \vspace{3mm}
  \begin{teorema}[Lov\'asz 1972]
    Sia \(G\) un grafo. Allora \(G\) \`e perfetto se e soltanto se \(\overline{G}\) \`e perfetto.
  \end{teorema}
\end{frame}

\begin{frame}
  {Relazione con gli insiemi parzialmente ordinati}
  \begin{definizione}
    Sia \(P = (V, \le_P)\) un insieme parzialmente ordinato. Chiamiamo 
    \emph{grafo di confrontabilità} il grafo \(G(P)\) di insieme di vertici 
    \(V\) e un arco fra \(v_i\,\text{,}v_j\in V\) se sono confrontabili 
    secondo \(\le_P\).
  \end{definizione}
  \vspace{3mm}
  Scriviamo \(H(\overline{P})\) per indicare \(H(\overline{G}(P))\). Vale:
  \vspace{3mm}
  \begin{teorema}[Kahn, Kim 1995]
    \[
      \log{e(P)}\le nH(\overline{P})\le c\log{e(P)}
    \]
  \end{teorema}
  dove \(c = 1 + 7\log{e}\approx 11.1\).
\end{frame}

\section{Gli algoritmi di Cardinal et al.}
\begin{frame}
  {``Merge sort na\"ive'' con informazione parziale}
  \begin{algorithm}[H]
    \caption{``Merge sort na\"ive'' con informazione parziale} 
    \begin{algorithmic}[1] 
      \STATE \, \COMMENT Preparazione 
      \STATE trova una decomposizione golosa di \(P\) in catene \(C_1,\dots,C_k\) 
      \STATE \(\mathcal{C}\leftarrow\left\{C_1,\dots,C_k\right\}\) 
      \STATE \, \COMMENT Ordinamento 
      \WHILE{\(|\mathcal{C}|>1\)} 
        \STATE seleziona da \(\mathcal{C}\) due catene di lunghezza minima \(C\) e \(C'\) 
        \STATE fondi \(C\) e \(C'\) in tempo lineare, ottenendo \(C''\) 
        \STATE cancella \(C\) e \(C'\) da \(\mathcal{C}\), aggiungi \(C''\)   
      \ENDWHILE 
      \RETURN l'unica catena di \(\mathcal{C}\) 
    \end{algorithmic}
  \end{algorithm}
\end{frame}

\begin{frame}
  {Esempio di ``Merge sort na\"ive''}
  \begin{figure}
    \centering
    \begin{tikzpicture}
      \node at (1,0) (a1_1) [vertex] {};
      \node at (0,1) (b1_1) [vertex] {};
      \node at (1,1) (a2_1) [vertex] {};
      \node at (2,1) (b2_1) [vertex] {};
      \node at (0,2) (a3_1) [vertex] {};
      \node at (1,2) (b3_1) [vertex] {};
      \node at (2,2) (b4_1) [vertex] {};
      \node at (1,3) (a4_1) [vertex] {};
      
      \draw [->] (a1_1) -- (a2_1);
      \draw [->] (a2_1) -- (a3_1);
      \draw [->] (a3_1) -- (a4_1);
      
      \draw [->] (a1_1) -- (b1_1);
      \draw [->] (a1_1) -- (b2_1);
      
      \draw [->] (b1_1) -- (b3_1);
      \draw [->] (b1_1) -- (b4_1);
      \draw [->] (a2_1) -- (b4_1);
      \draw [->] (b2_1) -- (a3_1);
      \draw [->] (b2_1) -- (b3_1);
      
      \draw [->] (b3_1) -- (a4_1);
      \draw [->] (b4_1) -- (a4_1);
      
      \path[use as bounding box] (-0.5,-0.5) rectangle (10,4);
    \end{tikzpicture}
  \end{figure}
\end{frame}

\begin{frame}
  {Esempio di ``Merge sort na\"ive''}
  \begin{figure}
    \centering
    \begin{tikzpicture}
      \node at (1,0) (a1) [vertex] {};
      \node at (0,1) (b1) [vertex] {};
      \node at (1,1) (a2) [vertex] {};
      \node at (2,1) (b2) [vertex] {};
      \node at (0,2) (a3) [vertex] {};
      \node at (1,2) (b3) [vertex] {};
      \node at (2,2) (b4) [vertex] {};
      \node at (1,3) (a4) [vertex] {};
      
      \draw [very thick, color=red] (a1) -- (a2);
      \draw [very thick, color=red] (a2) -- (a3);
      \draw [very thick, color=red] (a3) -- (a4);
      
      \draw (a1) -- (b1);
      \draw (a1) -- (b2);
      
      \draw (b1) -- (b3);
      \draw [very thick, color=blue] (b1) -- (b4);
      \draw (a2) -- (b4);
      \draw (b2) -- (a3);
      \draw [very thick, color=green] (b2) -- (b3);
      
      \draw (a4) -- (b3);
      \draw (a4) -- (b4);
      
      \path[use as bounding box] (-0.5,-0.5) rectangle (10,4);
    \end{tikzpicture}
  \end{figure}
\end{frame}

\begin{frame}
  {Esempio di ``Merge sort na\"ive''}
  \begin{figure}
    \centering
    \begin{tikzpicture}
      \node at (1,0) (a1) [vertex] {};
      \node at (0,1) (b1) [vertex] {};
      \node at (1,1) (a2) [vertex] {};
      \node at (2,1) (b2) [vertex] {};
      \node at (0,2) (a3) [vertex] {};
      \node at (1,2) (b3) [vertex] {};
      \node at (2,2) (b4) [vertex] {};
      \node at (1,3) (a4) [vertex] {};
      
      \draw [very thick, color=red] (a1) -- (a2);
      \draw [very thick, color=red] (a2) -- (a3);
      \draw [very thick, color=red] (a3) -- (a4);
      
      \draw (a1) -- (b1);
      \draw (a1) -- (b2);
      
      \draw (b1) -- (b3);
      \draw [very thick, color=blue] (b1) -- (b4);
      \draw (a2) -- (b4);
      \draw (b2) -- (a3);
      \draw [very thick, color=green] (b2) -- (b3);
      
      \draw (a4) -- (b3);
      \draw (a4) -- (b4);
      
      \node at (3, 3/2) {\(\longrightarrow\)};
      
      \node at (3.5,3) (c1_1) [vertex] {};
      \node at (4.5,3) (c1_2) [vertex] {};
      \node at (5.5,3) (c1_3) [vertex] {};
      \node at (6.5,3) (c1_4) [vertex] {};
      
      \node at (5.5,2) (c2_1) [vertex] {};
      \node at (6.5,2) (c2_2) [vertex] {};

      \node at (5.5,1) (c3_1) [vertex] {};
      \node at (6.5,1) (c3_2) [vertex] {};
      
      \draw [very thick, color=red] (c1_1) -- (c1_2);
      \draw [very thick, color=red] (c1_2) -- (c1_3);
      \draw [very thick, color=red] (c1_3) -- (c1_4);
      
      \draw [very thick, color=blue] (c2_1) -- (c2_2);
      
      \draw [very thick, color=green] (c3_1) -- (c3_2);
      
      \path[use as bounding box] (-0.5,-0.5) rectangle (10,4);
    \end{tikzpicture}
  \end{figure}
\end{frame}

\begin{frame}
  {Esempio di ``Merge sort na\"ive''}
  \begin{figure}
    \centering
    \begin{tikzpicture}
      \node at (1,0) (a1) [pale_vertex] {};
      \node at (0,1) (b1) [pale_vertex] {};
      \node at (1,1) (a2) [pale_vertex] {};
      \node at (2,1) (b2) [pale_vertex] {};
      \node at (0,2) (a3) [pale_vertex] {};
      \node at (1,2) (b3) [pale_vertex] {};
      \node at (2,2) (b4) [pale_vertex] {};
      \node at (1,3) (a4) [pale_vertex] {};
      
      \draw [very thick, color=red!40!white] (a1) -- (a2);
      \draw [very thick, color=red!40!white] (a2) -- (a3);
      \draw [very thick, color=red!40!white] (a3) -- (a4);
      
      \draw [color=black!25!white] (a1) -- (b1);
      \draw [color=black!25!white] (a1) -- (b2);
      
      \draw [color=black!25!white] (b1) -- (b3);
      \draw [very thick, color=blue!40!white] (b1) -- (b4);
      \draw [color=black!25!white] (a2) -- (b4);
      \draw [color=black!25!white] (b2) -- (a3);
      \draw [very thick, color=green!40!white] (b2) -- (b3);
      
      \draw [color=black!25!white] (a4_1) -- (b3_1);
      \draw [color=black!25!white] (a4_1) -- (b4_1);
      
      \node at (3.5,3) (c1_1) [vertex] {};
      \node at (4.5,3) (c1_2) [vertex] {};
      \node at (5.5,3) (c1_3) [vertex] {};
      \node at (6.5,3) (c1_4) [vertex] {};
      
      \node at (5.5,2) (c2_1) [vertex] {};
      \node at (6.5,2) (c2_2) [vertex] {};

      \node at (5.5,1) (c3_1) [vertex] {};
      \node at (6.5,1) (c3_2) [vertex] {};
      
      \draw [very thick, color=red] (c1_1) -- (c1_2);
      \draw [very thick, color=red] (c1_2) -- (c1_3);
      \draw [very thick, color=red] (c1_3) -- (c1_4);
      
      \draw [very thick, color=blue] (c2_1) -- (c2_2);
      
      \draw [very thick, color=green] (c3_1) -- (c3_2);
      
      \draw [thick] (7.5, 1) -- (8.5, 3/2);
      \draw [thick] (7.5, 2) -- (8.5, 3/2);
      
      \path[use as bounding box] (-0.5,-0.5) rectangle (10,4);
    \end{tikzpicture}
  \end{figure}
\end{frame}

\begin{frame}
  {Esempio di ``Merge sort na\"ive''}
  \begin{figure}
    \centering
    \begin{tikzpicture}
      \node at (1,0) (a1) [pale_vertex] {};
      \node at (0,1) (b1) [pale_vertex] {};
      \node at (1,1) (a2) [pale_vertex] {};
      \node at (2,1) (b2) [pale_vertex] {};
      \node at (0,2) (a3) [pale_vertex] {};
      \node at (1,2) (b3) [pale_vertex] {};
      \node at (2,2) (b4) [pale_vertex] {};
      \node at (1,3) (a4) [pale_vertex] {};
      
      \draw [very thick, color=red!40!white] (a1) -- (a2);
      \draw [very thick, color=red!40!white] (a2) -- (a3);
      \draw [very thick, color=red!40!white] (a3) -- (a4);
      
      \draw [color=black!25!white] (a1) -- (b1);
      \draw [color=black!25!white] (a1) -- (b2);
      
      \draw [color=black!25!white] (b1) -- (b3);
      \draw [very thick, color=blue!40!white] (b1) -- (b4);
      \draw [color=black!25!white] (a2) -- (b4);
      \draw [color=black!25!white] (b2) -- (a3);
      \draw [very thick, color=green!40!white] (b2) -- (b3);
      
      \draw [color=black!25!white] (a4_1) -- (b3_1);
      \draw [color=black!25!white] (a4_1) -- (b4_1);
      
      \node at (3.5,3) (c1_1) [vertex] {};
      \node at (4.5,3) (c1_2) [vertex] {};
      \node at (5.5,3) (c1_3) [vertex] {};
      \node at (6.5,3) (c1_4) [vertex] {};
      
      \node at (5.5,2) (c2_1) [vertex] {};
      \node at (6.5,2) (c2_2) [vertex] {};

      \node at (5.5,1) (c3_1) [vertex] {};
      \node at (6.5,1) (c3_2) [vertex] {};
      
      \draw [very thick, color=red] (c1_1) -- (c1_2);
      \draw [very thick, color=red] (c1_2) -- (c1_3);
      \draw [very thick, color=red] (c1_3) -- (c1_4);
      
      \draw [very thick, color=blue] (c2_1) -- (c2_2);
      
      \draw [very thick, color=green] (c3_1) -- (c3_2);
      
      \draw [thick] (7.5, 1) -- (8.5, 3/2);
      \draw [thick] (7.5, 2) -- (8.5, 3/2);
      
      \draw [thick] (8.5, 3/2) -- (9.5, 2);
      \draw [thick] (7.5, 3) -- (9.5, 2);
      
      \path[use as bounding box] (-0.5,-0.5) rectangle (10,4);
    \end{tikzpicture}
  \end{figure}
\end{frame}

\begin{frame}
  {Numero di confronti del ``Merge sort na\"ive''}
  \begin{proposizione}
    Sia \(P\) un insieme parzialmente ordinato di cardinalità \(n\). Allora, 
    per ogni \(\varepsilon>0\), l'algoritmo ``Merge sort na\"ive'' risolve il 
    problema dell'ordinamento parziale impiegando al più 
\[(1+\varepsilon)\log{e(P)}+(1+\varepsilon)\left(\log{e}+\log{\left(1+\frac{1}{\varepsilon}\right)}+1\right)\cdot n\] 
    confronti.
  \end{proposizione}
\end{frame}

\begin{frame}
  {Il problema della fusione con informazione parziale}
  \begin{definizione}
    Chiamiamo \emph{problema della fusione con informazione parziale} il caso 
    particolare del problema dell'ordinamento con informazione parziale in cui 
    \(P\) è un insieme parzialmente ordinato partizionabile in due catene 
    disgiunte \(A\) e \(B\).
  \end{definizione}
  \vspace{5mm}
  \begin{proposizione}
    Esiste un algoritmo (``Merge'') che risolva il problema della fusione con informazione 
    parziale impiegando al più \(6\log{e(P)}\) confronti.
  \end{proposizione}
\end{frame}

\begin{frame}
  {Il Teorema di Cardinal et al.}
  \begin{teorema}[Cardinal, Fiorini, Joret, Jungers, Munro 2010]
    Sia \(P\) un insieme parzialmente ordinato. Esiste un algoritmo che risolve il problema dell'ordinamento con informazione parziale impiegando al pi\`u \(c\log{e(P)}\) confronti e un numero polinomiale di operazioni elementari, dove \(c\approx 15.08\).
  \end{teorema}
\end{frame}

\begin{frame}
  {``Merge sort'' con informazione parziale}
  \begin{algorithm}[H]
    \caption{``Merge sort'' con informazione parziale} 
    \begin{algorithmic}[1] 
      \STATE trova una catena \(A\) di lunghezza massima in \(P\) 
      \STATE applica l'algoritmo ``Merge sort na\"ive'' a \(P-A\), ottieni la catena \(B\) 
      \STATE applica l'algoritmo ``Merge'' all'ordine parziale corrente \(P'\) 
      \RETURN la catena risultante 
    \end{algorithmic}
  \end{algorithm}
\end{frame}

\begin{frame}
  {Dimostrazione del Teorema di Cardinal et al. 1/2}
  \begin{dimostrazione}
    \begin{itemize}
    \item Trovare una catena di lunghezza massima non richiede
      confronti.
    \item Sappiamo che l'algoritmo ``Merge sort na\"ive'' impiega al
      più
      \[(1+\varepsilon)\log{e(P-A)}+(1+\varepsilon)\left(\log{e}+\log{\left(1+\frac{1}{\varepsilon}\right)}+1\right)\cdot
      |P-A|\] confronti \(\;\forall\,\varepsilon > 0\).
    \item L'ultima fusione comporta al più \(6\log{e(P')}\) confronti.
    \end{itemize}
  \end{dimostrazione}
\end{frame}

\begin{frame}
  {Dimostrazione del Teorema di Cardinal et al. 2/2}
  \begin{dimostrazione}
    Con qualche passaggio algebrico otteniamo che il numero di
    confronti effettuato dall'algoritmo ``Merge sort'' è maggiorato,
    \(\;\forall\,\varepsilon > 0\), da
    \[\left(1+\varepsilon+2\left((1+\varepsilon)\left(1+\ln{\left(1+\frac{1}{\varepsilon}\right)}\right)+\ln{2}\right)+6\right)\log{e(P)}.\]
    Basta allora porre \(\varepsilon\approx 0.351198\) e otteniamo la
    tesi.
  \end{dimostrazione}
\end{frame}

\begin{frame}
  {Ringraziamenti}
  \begin{center}
    Grazie dell'attenzione!
  \end{center}
\end{frame}

\end{document}
