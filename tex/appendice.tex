\appendix 
\chapter{Fine della dimostrazione del Teorema \ref{cfjjmtheorem}} 
\begin{proof}
  Resta da dimostrare la disuguaglianza \eqref{eq:appendixlemma}, cioè l'esistenza di un indice \(i\in\{1,2\}\) tale che
  \[\frac{e(P_i)}{e(P)}\le\frac{1}{\sqrt{\alpha_i\beta_i}}.\]
  Dimostreremo che 
  \begin{align}
    \frac{e(P_1)}{e(P)}+\frac{e(P_2)}{e(P)}&\le 1 \label{eq:extensions} \\
    \frac{1}{\sqrt{\alpha_1\beta_1}}+\frac{1}{\sqrt{\alpha_2\beta_2}}&\ge 1. \label{eq:inverseroots} 
  \end{align}
  Per dimostrare \eqref{eq:extensions} è sufficiente dimostrare che \(a<_{P_1} b\) e \(a>_{P_2} b\). Sfruttando le definizioni di \(\alpha_1\) e \(\beta_1\) abbiamo
  \[ y_{a^+}^1 = y_a+\frac{x_a}{\alpha_1} = 
  \begin{cases}
    y_{a^-}+x_a-x_b &\text{se}\quad\lambda\le\frac{1}{2}\\
    y_{a^-}+\frac{x_a}{2} &\text{altrimenti} 
  \end{cases}
  \]
  e
  \[ y_{b^-}^1 = y_{b^+}-\frac{x_b}{\beta_1} = y_a+x_a-\frac{x_b}{\beta_1} = 
  \begin{cases}
    y_{a^-}+x_a-x_b &\text{se}\quad\lambda\le\frac{1}{2} \\
    y_{a^-}+\frac{x_a}{2} &\text{altrimenti,} 
  \end{cases}
  \]
  cioè \(a<_{P_1} b\). Per le analoghe definizioni di \(\alpha_2\) e \(\beta_2\) abbiamo
  \[ y_{a^-}^2 = y_{a^+}-\frac{x_a}{\alpha_2} = y_{a^+} - \frac{x_b}{2} \]
  e
  \[ y_{b^+}^2 = y_{b^-}+\frac{x_b}{\beta_2} = y_{a^+}-x_b+\frac{x_b}{\beta_2} = y_{a^+}-\frac{x_b}{2}, \]
  cioè \(a>_{P_2} b\). Poiché \(P_1\) e \(P_2\) sono estensioni di \(P\) otteniamo la tesi.
  
  Per dimostrare \eqref{eq:inverseroots} basta studiare la funzione di \(\lambda\) definita dal membro sinistro. Abbiamo infatti
  \[ f(\lambda) = 
  \begin{cases}
    \sqrt{1-\lambda}+\frac{\sqrt{\lambda}}{2} &\text{se}\quad\lambda\le\frac{1}{2} \\
    \frac{1}{2\sqrt{\lambda}}+\frac{\sqrt{\lambda}}{2} &\text{altrimenti,} 
  \end{cases}
  \]
  la cui derivata è
  \[ f'(\lambda) = 
  \begin{cases}
    \frac{1}{4\sqrt{\lambda}}-\frac{1}{2\sqrt{1-\lambda}} &\text{se}\quad\lambda\le\frac{1}{2} \\
    \frac{1}{4\sqrt{\lambda}}-\frac{1}{4\lambda^{\frac{3}{2}}} &\text{altrimenti.} 
  \end{cases}
  \]
  \`E facile inoltre verificare che la derivata sia positiva per \(\lambda < \frac{1}{5}\) e negativa altrimenti. Poiché \(f(0)=f(1)=1\) possiamo dedurre che \(f\) è maggiore di \(1\) per ogni \(\lambda\in [0,1]\), cioè la tesi.\qed 
\end{proof}
