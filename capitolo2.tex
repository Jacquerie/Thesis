\documentclass[12pt]{article}

% To prevent nasty underfull boxes
\usepackage[italian]{babel}

% Use utf-8 encoding for foreign characters
\usepackage[utf8]{inputenc}

% Setup for fullpage use
\usepackage{fullpage}

% Uncomment some of the following if you use the features
%
% Running Headers and footers
%\usepackage{fancyhdr}
% Multipart figures
%\usepackage{subfigure}
% More symbols
\usepackage{amsmath} 
\usepackage{amssymb} 
\usepackage{latexsym}

% Surround parts of graphics with box
\usepackage{boxedminipage}

% Package for including code in the document
\usepackage{listings}

% If you want to generate a toc for each chapter (use with book)
\usepackage{minitoc}

% This is now the recommended way for checking for PDFLaTeX:
\usepackage{ifpdf}

%\newif\ifpdf
%\ifx\pdfoutput\undefined
%\pdffalse % we are not running PDFLaTeX
%\else
%\pdfoutput=1 % we are running PDFLaTeX
%\pdftrue
%\fi
\newtheorem{theorem}{Teorema}[section] 
\newtheorem{lemma}[theorem]{Lemma} 
\newtheorem{proposition}[theorem]{Proposizione} 
\newtheorem{corollary}[theorem]{Corollario}

\newenvironment{proof}[1][Dimostrazione.]{ 
\begin{trivlist}
	\item[\hskip \labelsep {\bfseries #1}]}{ 
\end{trivlist}
} 
\newenvironment{definition}[1][Definizione.]{ 
\begin{trivlist}
	\item[\hskip \labelsep {\bfseries #1}]}{ 
\end{trivlist}
} 
\newenvironment{example}[1][Esempio.]{ 
\begin{trivlist}
	\item[\hskip \labelsep {\bfseries #1}]}{ 
\end{trivlist}
} 
\newenvironment{remark}[1][Osservazione.]{ 
\begin{trivlist}
	\item[\hskip \labelsep {\bfseries #1}]}{ 
\end{trivlist}
}

% Black QED tombstone
% \newcommand{\qed}{\nobreak \ifvmode \relax \else
%       \ifdim\lastskip<1.5em \hskip-\lastskip
%       \hskip1.5em plus0em minus0.5em \fi \nobreak
%       \vrule height0.75em width0.5em depth0.25em\fi}
% White QED box
\newcommand{\qed}{\hfill \ensuremath{\Box}}

\ifpdf 
\usepackage[pdftex]{graphicx} \else 
\usepackage{graphicx} \fi 
\title{Proprietà e teoremi notevoli} 
\author{Jacopo Notarstefano}

\begin{document}

\ifpdf \DeclareGraphicsExtensions{.pdf, .jpg, .tif} \else \DeclareGraphicsExtensions{.eps, .jpg} \fi

\maketitle

\section{Proprietà notevoli dell'entropia di grafo}

\subsection{Monotonia} 
\begin{lemma}
	Siano \(F\) e \(G\) grafi tali che \(V(G)=V(F)\) ma \(E(F)\subset E(G)\). Allora comunque scelta \(P\) densità discreta sui vertici avremo \(H(F,P)\le H(G,P)\). 
\end{lemma}
\begin{proof}
	Osserviamo che se \(E(F)\subset E(G)\) allora \(\text{VP}(G)\subset \text{VP}(F)\). Sfruttando la terza definizione di entropia di grafo abbiamo immediatamente la tesi, infatti stiamo prendendo il minimo della stessa funzione obiettivo su un insieme più grande.\qed 
\end{proof}

\subsection{Subadditività} 
\begin{lemma}
	Siano \(F\) e \(G\) grafi di comune insieme dei vertici \(V\). Sia \(F\cup G\) il grafo di vertici \(V\) ed insieme degli archi \(E(F)\cup E(G)\). Comunque scelta \(P\) densità discreta sui vertici avremo
	\[H(F\cup G,P)\le H(F,P)+H(G,P)\]
\end{lemma}
\begin{proof}
	Siano \(\mathbf{a}\in \text{VP}(F)\) e \(\mathbf{b}\in \text{VP}(G)\) i vettori che realizzino il minimo delle rispettive entropie. Osserviamo che l'intersezione di un insieme indipendente di \(F\) e di un insieme indipendente di \(G\) è un insieme indipendente in \(F\cup G\). In altri termini il prodotto scalare dei loro vettori caratteristici è il vettore caratteristico di un insieme indipendente di \(F\cup G\). Pertanto, sfruttando la convessità del politopo dei vertici, il prodotto scalare \(\mathbf{a}\cdot \mathbf{b}\) appartiene a \(\text{VP}(F\cup G)\). Ma allora possiamo scrivere
	\[H(F,P)+H(G,P)=\sum_{i=1}^n p_i\log{\frac{1}{a_i}}+\sum_{i=1}^n p_i\log{\frac{1}{b_i}}=\sum_{i=1}^n p_i\log{\frac{1}{a_{i}b_{i}}}\ge H(F\cup G,P)\]
	\qed 
\end{proof}

\subsection{Additività per sostituzioni}
Siano \(F\) e \(G\) grafi su insiemi di vertici disgiunti, sia \(v\) un vertice di \(G\). Chiamiamo grafo ottenuto sostituendo \(F\) a \(v\), e scriviamo \(G_{v\leftarrow F}\), il grafo ottenuto da \(G\) cancellando \(v\) e connettendo ogni vertice adiacente a \(v\) con ogni vertice di una copia isomorfa di \(F\). Supponiamo inoltre che \(P\) sia una densità sui vertici di \(G\) e che \(Q\) sia una densità sui vertici di \(F\). Allora possiamo definire una \(P_{v\leftarrow Q}\) in modo che la coppia \((G_{v\leftarrow F}, P_{v\leftarrow Q})\) sia un grafo probabilistico. Per fare questo poniamo
\[P_{v\leftarrow Q}(x)= 
\begin{cases}
	P(x) & \text{se}\ x\in V(G)-\{v\}\\
	P(v)Q(x) & \text{se}\ x\in V(F) 
\end{cases}
\]
\begin{lemma}
	Siano \(F\) e \(G\) grafi su insiemi di vertici disgiunti, sia \(v\) un vertice di \(G\). Siano inoltre \(P\) una densità sui vertici di \(G\) e \(Q\) una densità sui vertici di \(F\). Allora abbiamo
	\[H(G_{v\leftarrow F}, P_{v\leftarrow Q})=H(G,P)+P(v)H(F,Q)\]
\end{lemma}
\begin{proof}
	TODO\qed 
\end{proof}
\begin{corollary}
	Sia \((G,P)\) un grafo probabilistico e siano \(G_{1}\dots G_{n}\) le sue componenti connesse. Poniamo \(P_i(x)=P(x)[P(V(G_i))]^{-1}\) per \(x\in V(G_i)\). Allora abbiamo
	\[H(G,P)=\sum_{i=1}^n P(V(G_i))H(G_i,P_i)\]
\end{corollary}
\begin{proof}
	Consideriamo il grafo su \(n\) vertici \(\{v_{1}\dots v_{n}\}\) privo di archi, e sia \(Q\) la densità discreta tale che \(Q(v_{i}) = P(V(G_i))\). Otteniamo la tesi applicando \(n\) volte il lemma precedente, sostituendo ad ogni passo il vertice \(v_i\) con la componente connessa \(G_{i}\).\qed 
\end{proof}

\subsection{Entropia di grafo completo} 
\begin{proposition}
	Sia \(K_n\) il grafo completo su \(n\) vertici. Comunque scelta \(P\) densità discreta sui vertici avremo
	\[H(K_n,P)=H(P)\]
\end{proposition}
\begin{proof}
	Sfruttando la terza definizione di entropia di grafo sappiamo che
	\[H(K_n,P)=\sum_{i=1}^n p_i \log{\frac{1}{q_i}}\]
	per certi \(q_1\dots q_n\) positivi. Osserviamo inoltre che nel grafo completo gli insiemi indipendenti sono soltanto \(\varnothing\) e i singoletti dei vertici. Pertanto il politopo dei vertici è l'\(n-\)simplesso, ma poiché sappiamo che la funzione obiettivo è minima sul bordo deduciamo che
	\[\sum_{i=1}^n q_i = 1.\]
	Applichiamo ora la diseguaglianza sulle somme dei logaritmi ed otteniamo che il minimo è realizzato per
	\[q_i = p_i\quad \forall i\]
	\qed 
\end{proof}
\begin{remark}
	Come anticipato nell'introduzione abbiamo riottenuto l'entropia di Shannon come caso particolare dell'entropia di grafo. 
\end{remark}

% ++++++++++++++++++++++++++++++++++++++++++++++++++++++++++++++++++++++++++ %
\section{Entropia e grafi perfetti} Il numero di cricca di un grafo \(G\) fornisce una stima dal basso del numero cromatico. È infatti evidente che \(\omega(G)\le\chi(G)\), poiché avremo bisogno di almeno tanti colori quanti sono i vertici della massima cricca. Possiamo dunque porci il problema di caratterizzare quei grafi per cui tale disuguaglianza sia in realtà una uguaglianza. 
\begin{definition}
	Sia \(G\) un grafo. Diciamo che \(G\) è \emph{perfetto} se, per ogni sottografo \(H\), vale
	\[\omega(H)=\chi(H).\]
\end{definition}
Nella precedente definizione abbiamo richiesto che l'uguaglianza valga per ogni sottografo al fine di non considerare perfette le unioni disgiunte di componenti per cui valga l'uguaglianza e componenti per cui non valga. Esiste un sorprendente collegamento fra l'entropia di grafo e i grafi perfetti. 
\begin{theorem}
	[Csiszár, K\"orner, Lovász, Marton, Simonyi] Sia \(G\) un grafo. \(G\) è perfetto se e soltanto se, per ogni distribuzione di probabilità \(P\) sui vertici, vale
	\[H\left(G,P\right)+H\left(\overline{G},P\right)=H(P).\]
\end{theorem}
Come corollario del precedente teorema otteniamo che \(G\) è perfetto se e soltanto se \(\overline{G}\) è perfetto.
  

% ++++++++++++++++++++++++++++++++++++++++++++++++++++++++++++++++++++++++++ %
\section{Grafi associati ad ordini parziali}

Sia \(G\) un grafo. Possiamo trovare una partizione dei suoi vertici in insiemi indipendenti con l'algoritmo goloso che ad ogni passo trova un insieme indipendente e massimale e procede ricorsivamente sul complementare. 
\begin{definition}
	Sia \(G\) un grafo perfetto e sia \(\left\{S_1,\dots,S_k\right\}\) una partizione dei vertici ottenuta con il precedente algoritmo goloso. Chiameremo \emph{punto goloso} il punto \(x\) definito da
	\[x=\sum_{i=1}^k\frac{|S_i|}{n}\chi^{S_i}\]
\end{definition}
È evidente per costruzione che tale punto appartenga a \(\text{STAB}(G)\). Il seguente teorema
\begin{theorem}[Cardinal, Fiorini, Joret, Jungers, Munro]
	Sia \(G\) un grafo perfetto su \(n\) vertici e sia \(x\) un suo punto goloso. Allora, comunque fissato \(\varepsilon>0\), vale
	\[H(x)\le(1+\varepsilon)H(G)+(1+\varepsilon)\log\left(1+\frac{1}{\varepsilon}\right)\]
\end{theorem}
\begin{proof}
	Sia \(S_1,\dots,S_k\) la sequenza di insiemi indipendenti prodotta dall'algoritmo goloso. In altri termini \(S_1\) è un insieme indipendente e massimale in \(G\), mentre \(S_2\) è indipendente e massimale in \(G-S_1\) e così via. Sia \(\delta>0\) fissato. Per ogni vertice \(v\in V\) denotiamo con \(m(v)\) l'unico indice in \(\left\{1,\dots,k\right\}\) tale che \(v\in S_m(v)\). Definiamo allora un punto \(z\) di componenti date da
	\[z_v=\frac{\delta}{n}\left(\frac{1}{x_v}\right)^{1-\delta}=\frac{\delta}{n}\left(\frac{n}{|S_{m(v)}|}\right)^{1-\delta}=\frac{\delta}{n^{\delta}}\left(\frac{1}{|S_{m(v)}|}\right)^{1-\delta}\]
	e dimostriamo che \(z\in\text{STAB}(G)\). A tale scopo mostreremo che, per ogni insieme indipendente \(S\), vale
	\[\sum_{v\in S}{z_v}\le 1\]
	
	Possiamo ora concludere.
	\begin{eqnarray}
	  H(G)&=&\log(n)-H(\overline{G}) \nonumber \\
	      &\ge& \log(n) + \frac{1}{n}\sum_{v\in V}{\log{z_v}}\nonumber \\
	      &=& \log(n) + \frac{1}{n}\sum_{v\in V}{\log\left(\frac{\delta}{n}\left(\frac{1}{x_v}\right)^{1-\delta}\right)} \nonumber \\
	      &=& - \frac{1-\delta}{n}\sum_{v\in V}{\log(x_v)}-\log{\frac{1}{\delta}} \nonumber \\
	      &=& (1-\delta)H(x)-\log{\frac{1}{\delta}} \nonumber
	\end{eqnarray}
\end{proof}

\end{document} 
