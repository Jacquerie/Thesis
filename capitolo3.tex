\chapter{Tre algoritmi per ordinare con informazione parziale}

% ++++++++++++++++++++++++++++++++++++++++++++++++++++++++++++++++++++++++++ %
\section{Ordinamento con informazione parziale}
Sia \(V=\left\{v_1,\dots,v_n\right\}\) un insieme dotato di un ordine totale \(\le\) fissato ma ignoto. Supponiamo di conoscere un sottoinsieme delle relazioni del tipo \(v_i\le v_j\), e di voler determinare l'ordine totale per mezzo di interrogazioni del tipo \(``\text{\emph{è vero che}}\;v_i\le v_j\;\text{?}\,''\).

\begin{theorem}[Cardinal, Fiorini, Joret, Jungers, Munro]
  Sia \(P\) un insieme parzialmente ordinato di cardinalità \(n\). Allora vale
  \[nH\left(\overline{P}\right)\le 2\log{e(P)}.\]
\end{theorem}
\begin{proof}
  La dimostrazione procede per induzione su \(n\), e per \(n\) fissato, sul numero di elementi inconfrontabili di \(P\). Essendo la tesi banalmente vera per \(n=1\) supporremo che sia \(n\ge 2\). Sia \(x\in \mathbb{R}_{+}^{V}\) un vettore che realizzi il minimo dell'entropia. Sia inoltre \(\left\{\left(y_{v^-},y_{v^+}\right)\right\}_{v\in V}\) la corrispondente collezione di intervalli. Sia infine \(a\in V\) tale che \(y_{a^+}\) sia massimo.
  Se \(a\) fosse confrontabile con tutti gli elementi di \(V\) avremmo per ipotesi induttiva che
  \[nH\left(\overline{P}\right)=(n-1)H\left(\overline{P-a}\right)\le 2\log{e(P-a)=2\log{e(P)}}.\]
  Sia allora \(b\) non confrontabile con \(a\) e tale inoltre che \(y_{b^+}\) sia massimo. Per come abbiamo scelto \(a\) deve per forza essere \(y_{b^+}\le y_{a^+}\). In realtà vale l'uguaglianza. Supponiamo per assurdo che \(y_{b^+}<y_{a^+}\), ed estendiamo l'intervallo corrispondente a \(b\) a destra di \(y_{a^+}-y_{b^+}\). Questa nuova collezione di intervalli è ancora consistente con \(P\), ma il punto \(x'\in \mathbb{R}_{+}^{V}\) da essa definito realizza un valore dell'entropia più piccolo del minimo. Abbiamo infatti
  \[-\frac{1}{n}\sum_{v\in V}{\log{x'_{v}}}=-\frac{1}{n}\sum_{v\in V}{\log{x_{v}}}+\frac{1}{n}\left(\log{x_b}-\log{x'_{b}}\right)<-\frac{1}{n}\sum_{v\in V}{\log{x_v}},\]
  contro l'ipotesi che \(x\) realizzi il minimo dell'entropia.
  A meno di scambiare \(a\) e \(b\) possiamo ora supporre che \(x_a\ge x_b\)
\end{proof}

% ++++++++++++++++++++++++++++++++++++++++++++++++++++++++++++++++++++++++++ %
\section{Insertion sort} 
\begin{lemma}
	Sia \(P\) un insieme parzialmente ordinato di cardinalità \(n\) e sia \(C\) una catena di lunghezza massima in \(P\). Vale allora \(\left|C\right|\ge n\cdot2^{-H(\overline{P})}\). 
\end{lemma}

% ++++++++++++++++++++++++++++++++++++++++++++++++++++++++++++++++++++++++++ %
\section{Merge sort naive}

% ++++++++++++++++++++++++++++++++++++++++++++++++++++++++++++++++++++++++++ %
\section{Merge con informazione parziale} 
\begin{definition}
	Diremo che un arco \(uv\) è \emph{stretto} rispetto ad \(x\in\text{STAB}(G)\) se vale \(x_u+x_v=1\). Denoteremo con \(G(x)\) il grafo i cui vertici siano gli stessi di \(G\) e i cui archi siano stretti rispetto ad \(x\). 
\end{definition}
\begin{definition}
	Siano \(uv\) e \(u'v'\) archi di \(G\) tali che \(u,u'\in A\) e \(v,v'\in B\). Diremo che \emph{si incrociano} se \(u<_{P}u'\) e \(v'<_{P}v\) oppure se \(u'<_{P}u\) e \(v<_{P}v'\). 
\end{definition}
\begin{lemma}
	Sia \(P\) un insieme parzialmente ordinato ricoperto da due catene disgiunte \(A\) e \(B\) e sia \(G=\overline{G}(P)\) il suo grafo associato. Sia \(x\) un punto di \(\text{STAB}(G)\) e siano \(uv\) e \(u'v'\) archi stretti rispetto ad \(x\) tali inoltre che \(u,u'\in A\) e \(v,v'\in B\). Se \(uv\) e \(u'v'\) si incrociano allora sia \(u'v\) sia \(uv'\) sono archi di \(G\), entrambi stretti rispetto ad \(x\). 
\end{lemma}
\begin{proof}
	TODO 
\end{proof}
\begin{definition}
	Diremo che \(x\in\text{STAB}(G)\) è \emph{localmente ottimo} se per ogni componente connessa \(K\) di \(G(x)\) valgono
	\[x_u=\frac{|A\cap K|}{|K|}\;\text{per ogni}\,u\in A\cap K\qquad\text{e}\qquad x_v=\frac{|B\cap K}{|K|}\;\text{per ogni}\,v\in B\cap K.\]
	Diremo che \(K\) è \emph{bilanciata} se per essa valgono le precedenti condizioni di ottimalità, \emph{sbilanciata} altrimenti. 
\end{definition}
\begin{lemma}
	Siano \(X\) e \(Y\) due catene disgiunte. Supponiamo che \(|X|\ge|Y|\). Allora il numero di confronti richiesto dall'algoritmo di Hwang-Lin è maggiorato da \(|Y|\log(\frac{4|X|}{|Y|})\). 
\end{lemma}
\begin{proof}
	È noto che l'algoritmo di Hwang-Lin compia al più
	\[|Y|\left(1+\left\lfloor{\log{\frac{X}{Y}}}\right\rfloor\right)+\left\lfloor\frac{|X|}{2^{\left\lfloor\log{\frac{|X|}{|Y|}}\right\rfloor}}\right\rfloor-1\]
	confronti. Sia allora \(\xi\in\left[0,1\right)\) tale che
	\[\left\lfloor\log{\frac{|X|}{|Y|}}\right\rfloor=\log{\frac{|X|}{|Y|}}-\xi.\]
	È facile verificare che per \(\xi\in\left[0,1\right)\) valga la disuguaglianza
	\[ 1-\xi+2^{\xi}\le 2. \]
	Passaggi algebrici altrettanto facili danno
	\[ \frac{|X|}{2^{\left\lfloor\log{\frac{|X|}{|Y|}}\right\rfloor}}=\frac{|X|}{2^{\log{\frac{|X|}{|Y|}}-\xi}}=\frac{|X|}{2^{\log{\frac{|X|}{|Y|}}}}\cdot 2^{\xi}=|Y|\cdot 2^{\xi}. \]
	Possiamo infine mettere insieme le precedenti due equazioni per ottenere 
	\begin{eqnarray}
		|Y|\left(1+\left\lfloor{\log{\frac{X}{Y}}}\right\rfloor\right)+\left\lfloor\frac{|X|}{2^{\left\lfloor\log{\frac{|X|}{|Y|}}\right\rfloor}}\right\rfloor-1&\le&|Y|\left(1-\xi+\log{\frac{|X|}{|Y|}}+2^{\xi}\right) \nonumber \\
		&\le& |Y|\left(\log{\frac{|X|}{|Y|}}+2\right) \nonumber \\
		&=& |Y|\left(\log{\frac{4|X|}{|Y|}}\right) \nonumber 
	\end{eqnarray}
	cioé la tesi.\qed
\end{proof}
\begin{definition}
	Sia \(K\) una componente connessa di \(G(x)\). Se \(K\) è rossa chiamo \(A\cap K\) \emph{catena maggiore} e \(B\cap K\) \emph{catena minore}. Se \(K\) è blu il contrario. 
\end{definition}
\begin{definition}
	Sia \(K\) una componente connessa di \(G(x)\). Dico che \(K\) è \emph{buona} se ogni arco di \(G\) che possiede un estremo nella catena minore di \(K\) ha l'altro estremo nella catena maggiore oppure in una componente connessa di colore opposto. 
\end{definition}
\begin{lemma}
	Sia \(x\in \text{STAB}(G)\) localmente ottimo. Se \(G(x)\) possiede almeno una componente rossa non banale allora una di esse è buona. 
\end{lemma}
\begin{proof}
	Sia \(K\) una componente connessa rossa non banale tale che \(\frac{|A\cap K|}{|K|}\) sia minimo. Vogliamo dimostrare che \(K\) è buona. Sia \(v\in B\cap K\) e sia \(w\) adiacente a \(v\) in \(G\) ma non in \(G(x)\). Per definizione l'arco di estremi \(v\) e \(w\) non è stretto, quindi \(x_v+x_w<1\). In particolare \(x_w<1\), quindi \(w\) appartiene ad una qualche componente connessa \(L\) non banale. Se per assurdo \(L\) fosse rossa per ipotesi \(\frac{|A\cap L|}{|L|}\ge\frac{|A\cap K|}{|K|}\), dunque per ottimalità di \(x\) avremo
	\[x_v+x_w=\frac{|B\cap K|}{|K|}+\frac{|A\cap L|}{|L|}\ge\frac{|B\cap K|}{|K|}+\frac{|A\cap K|}{|K|}\ge 1\]
	da cui dedurremmo che l'arco di estremi \(v\) e \(w\) è stretto, una contraddizione. Segue quindi che \(L\) è blu oppure non esiste \(w\) adiacente a \(v\) in \(G\) ma non in \(G(x)\), cioè la tesi.\qed
\end{proof}

% ++++++++++++++++++++++++++++++++++++++++++++++++++++++++++++++++++++++++++ %
\section{Merge sort} 
