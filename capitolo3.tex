\chapter{Tre algoritmi per ordinare con informazione parziale}

% ++++++++++++++++++++++++++++++++++++++++++++++++++++++++++++++++++++++++++ %
\section{Ordinamento con informazione parziale}

% ++++++++++++++++++++++++++++++++++++++++++++++++++++++++++++++++++++++++++ %
\section{Insertion sort} 
\begin{lemma}
	Sia \(P\) un insieme parzialmente ordinato di cardinalità \(n\) e sia \(C\) una catena di lunghezza massima in \(P\). Vale allora \(\left|C\right|\ge n\cdot2^{-H(\overline{P})}\). 
\end{lemma}

% ++++++++++++++++++++++++++++++++++++++++++++++++++++++++++++++++++++++++++ %
\section{Merge sort naive}

% ++++++++++++++++++++++++++++++++++++++++++++++++++++++++++++++++++++++++++ %
\section{Merge con informazione parziale}

% ++++++++++++++++++++++++++++++++++++++++++++++++++++++++++++++++++++++++++ %
\section{Merge sort} 
\begin{definition}
	Sia \(K\) una componente connessa di \(G(x)\). Se \(K\) è rossa chiamo \(A\cap K\) \emph{catena maggiore} e \(B\cap K\) \emph{catena minore}. Se \(K\) è blu il contrario. 
\end{definition}
\begin{definition}
	Sia \(K\) una componente connessa di \(G(x)\). Dico che \(K\) è \emph{buona} se ogni arco di \(G\) che possiede un estremo nella catena minore di \(K\) ha l'altro estremo nella catena maggiore oppure in una componente connessa di colore opposto. 
\end{definition}
\begin{lemma}
	Sia \(x\in \text{STAB}(G)\) localmente ottimo. Se \(G(x)\) possiede almeno una componente rossa non banale allora una di esse è buona. 
\end{lemma}
\begin{proof}
	Sia \(K\) una componente connessa rossa non banale tale che \(\frac{|A\cap K|}{|K|}\) sia minimo. Vogliamo dimostrare che \(K\) è buona. Sia \(v\in B\cap K\) e sia \(w\) adiacente a \(v\) in \(G\) ma non in \(G(x)\). Per definizione l'arco di estremi \(v\) e \(w\) non è stretto, quindi \(x_v+x_w<1\). In particolare \(x_w<1\), quindi \(w\) appartiene ad una qualche componente connessa \(L\) non banale. Se per assurdo \(L\) fosse rossa per ipotesi \(\frac{|A\cap L|}{|L|}\ge\frac{|A\cap K|}{|K|}\), dunque per ottimalità di \(x\) avremo
	\[x_v+x_w=\frac{|B\cap K|}{|K|}+\frac{|A\cap L|}{|L|}\ge\frac{|B\cap K|}{|K|}+\frac{|A\cap K|}{|K|}\ge 1\]
	da cui dedurremmo che l'arco di estremi \(v\) e \(w\) è stretto, una contraddizione. Segue quindi che \(L\) è blu oppure non esiste \(w\) adiacente a \(v\) in \(G\) ma non in \(G(x)\), cioè la tesi. 
\end{proof}