\chapter{Conclusioni}
Nel primo capitolo di questa tesi abbiamo presentato tre definizioni di una 
generalizzazione dell'entropia di Shannon nota come entropia di grafo. La 
prima di esse è motivata da un problema di codifica, ma risulta in 
un'espressione difficile da calcolare e non evidentemente ben definita. 
Abbiamo quindi dimostrato l'equivalenza con una seconda forma, comunque 
espressa in termini di teoria dell'informazione, per la quale risulta facile 
affermare la buona definizione. È infine immediato dimostrare l'equivalenza 
con una terza, più utile nella pratica, che consente di vedere il calcolo 
dell'entropia come un problema di minimizzazione di una funzione convessa su 
un convesso.

Nel secondo capitolo sono state presentate le proprietà principali, fra cui
una forma di subadditività per coppie di grafi aventi lo stesso insieme di
vertici. Abbiamo evidenziato il sorprendente collegamento esistente fra
l'entropia e i grafi perfetti, una vasta classe di grafi dalle interessanti
proprietà algoritmiche e combinatoriali. Sono in particolare perfetti i grafi
di confrontabilità associati a ordini parziali; questo ci ha consentito di
riformulare l'entropia di grafo in termini puramente combinatoriali e di
ottenere un utile risultato sulla propria approssimabilità.

Il terzo capitolo di questa tesi ha applicato i precedenti risultati alla
soluzione del problema dell'ordinamento con informazione parziale. Tale
problema consiste nell'estensione di un ordine parziale a un ordine totale
facendo uso del minimo numero di ulteriori confronti. L'entropia di grafo,
come mostrato da Kahn e Kim, consente di esibire algoritmi ottimi per questo 
problema. Abbiamo presentato in particolare tre algoritmi dovuti a Cardinal et
al. che, sfruttando la versione approssimata dell'entropia presentata nel
precedente capitolo, riescono a evitare l'uso del metodo dell'ellissoide,
essenziale invece nell'articolo di Kahn e Kim.
